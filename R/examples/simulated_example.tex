% Options for packages loaded elsewhere
\PassOptionsToPackage{unicode}{hyperref}
\PassOptionsToPackage{hyphens}{url}
%
\documentclass[
]{article}
\usepackage{amsmath,amssymb}
\usepackage{lmodern}
\usepackage{ifxetex,ifluatex}
\ifnum 0\ifxetex 1\fi\ifluatex 1\fi=0 % if pdftex
  \usepackage[T1]{fontenc}
  \usepackage[utf8]{inputenc}
  \usepackage{textcomp} % provide euro and other symbols
\else % if luatex or xetex
  \usepackage{unicode-math}
  \defaultfontfeatures{Scale=MatchLowercase}
  \defaultfontfeatures[\rmfamily]{Ligatures=TeX,Scale=1}
\fi
% Use upquote if available, for straight quotes in verbatim environments
\IfFileExists{upquote.sty}{\usepackage{upquote}}{}
\IfFileExists{microtype.sty}{% use microtype if available
  \usepackage[]{microtype}
  \UseMicrotypeSet[protrusion]{basicmath} % disable protrusion for tt fonts
}{}
\makeatletter
\@ifundefined{KOMAClassName}{% if non-KOMA class
  \IfFileExists{parskip.sty}{%
    \usepackage{parskip}
  }{% else
    \setlength{\parindent}{0pt}
    \setlength{\parskip}{6pt plus 2pt minus 1pt}}
}{% if KOMA class
  \KOMAoptions{parskip=half}}
\makeatother
\usepackage{xcolor}
\IfFileExists{xurl.sty}{\usepackage{xurl}}{} % add URL line breaks if available
\IfFileExists{bookmark.sty}{\usepackage{bookmark}}{\usepackage{hyperref}}
\hypersetup{
  pdftitle={Some Examples Using Simulation},
  pdfauthor={Ryan Lafferty},
  hidelinks,
  pdfcreator={LaTeX via pandoc}}
\urlstyle{same} % disable monospaced font for URLs
\usepackage[margin=1in]{geometry}
\usepackage{color}
\usepackage{fancyvrb}
\newcommand{\VerbBar}{|}
\newcommand{\VERB}{\Verb[commandchars=\\\{\}]}
\DefineVerbatimEnvironment{Highlighting}{Verbatim}{commandchars=\\\{\}}
% Add ',fontsize=\small' for more characters per line
\usepackage{framed}
\definecolor{shadecolor}{RGB}{248,248,248}
\newenvironment{Shaded}{\begin{snugshade}}{\end{snugshade}}
\newcommand{\AlertTok}[1]{\textcolor[rgb]{0.94,0.16,0.16}{#1}}
\newcommand{\AnnotationTok}[1]{\textcolor[rgb]{0.56,0.35,0.01}{\textbf{\textit{#1}}}}
\newcommand{\AttributeTok}[1]{\textcolor[rgb]{0.77,0.63,0.00}{#1}}
\newcommand{\BaseNTok}[1]{\textcolor[rgb]{0.00,0.00,0.81}{#1}}
\newcommand{\BuiltInTok}[1]{#1}
\newcommand{\CharTok}[1]{\textcolor[rgb]{0.31,0.60,0.02}{#1}}
\newcommand{\CommentTok}[1]{\textcolor[rgb]{0.56,0.35,0.01}{\textit{#1}}}
\newcommand{\CommentVarTok}[1]{\textcolor[rgb]{0.56,0.35,0.01}{\textbf{\textit{#1}}}}
\newcommand{\ConstantTok}[1]{\textcolor[rgb]{0.00,0.00,0.00}{#1}}
\newcommand{\ControlFlowTok}[1]{\textcolor[rgb]{0.13,0.29,0.53}{\textbf{#1}}}
\newcommand{\DataTypeTok}[1]{\textcolor[rgb]{0.13,0.29,0.53}{#1}}
\newcommand{\DecValTok}[1]{\textcolor[rgb]{0.00,0.00,0.81}{#1}}
\newcommand{\DocumentationTok}[1]{\textcolor[rgb]{0.56,0.35,0.01}{\textbf{\textit{#1}}}}
\newcommand{\ErrorTok}[1]{\textcolor[rgb]{0.64,0.00,0.00}{\textbf{#1}}}
\newcommand{\ExtensionTok}[1]{#1}
\newcommand{\FloatTok}[1]{\textcolor[rgb]{0.00,0.00,0.81}{#1}}
\newcommand{\FunctionTok}[1]{\textcolor[rgb]{0.00,0.00,0.00}{#1}}
\newcommand{\ImportTok}[1]{#1}
\newcommand{\InformationTok}[1]{\textcolor[rgb]{0.56,0.35,0.01}{\textbf{\textit{#1}}}}
\newcommand{\KeywordTok}[1]{\textcolor[rgb]{0.13,0.29,0.53}{\textbf{#1}}}
\newcommand{\NormalTok}[1]{#1}
\newcommand{\OperatorTok}[1]{\textcolor[rgb]{0.81,0.36,0.00}{\textbf{#1}}}
\newcommand{\OtherTok}[1]{\textcolor[rgb]{0.56,0.35,0.01}{#1}}
\newcommand{\PreprocessorTok}[1]{\textcolor[rgb]{0.56,0.35,0.01}{\textit{#1}}}
\newcommand{\RegionMarkerTok}[1]{#1}
\newcommand{\SpecialCharTok}[1]{\textcolor[rgb]{0.00,0.00,0.00}{#1}}
\newcommand{\SpecialStringTok}[1]{\textcolor[rgb]{0.31,0.60,0.02}{#1}}
\newcommand{\StringTok}[1]{\textcolor[rgb]{0.31,0.60,0.02}{#1}}
\newcommand{\VariableTok}[1]{\textcolor[rgb]{0.00,0.00,0.00}{#1}}
\newcommand{\VerbatimStringTok}[1]{\textcolor[rgb]{0.31,0.60,0.02}{#1}}
\newcommand{\WarningTok}[1]{\textcolor[rgb]{0.56,0.35,0.01}{\textbf{\textit{#1}}}}
\usepackage{graphicx}
\makeatletter
\def\maxwidth{\ifdim\Gin@nat@width>\linewidth\linewidth\else\Gin@nat@width\fi}
\def\maxheight{\ifdim\Gin@nat@height>\textheight\textheight\else\Gin@nat@height\fi}
\makeatother
% Scale images if necessary, so that they will not overflow the page
% margins by default, and it is still possible to overwrite the defaults
% using explicit options in \includegraphics[width, height, ...]{}
\setkeys{Gin}{width=\maxwidth,height=\maxheight,keepaspectratio}
% Set default figure placement to htbp
\makeatletter
\def\fps@figure{htbp}
\makeatother
\setlength{\emergencystretch}{3em} % prevent overfull lines
\providecommand{\tightlist}{%
  \setlength{\itemsep}{0pt}\setlength{\parskip}{0pt}}
\setcounter{secnumdepth}{-\maxdimen} % remove section numbering
\ifluatex
  \usepackage{selnolig}  % disable illegal ligatures
\fi

\title{Some Examples Using Simulation}
\author{Ryan Lafferty}
\date{2/28/2022}

\begin{document}
\maketitle

\hypertarget{simulated-racial-discrimination}{%
\subsubsection{Simulated Racial
Discrimination}\label{simulated-racial-discrimination}}

Let's try to simulate a situation where discrimination is happening and
see if the PB method can detect it. Suppose black and white applicants
are applying for a certain job. Imagine that it is permissible to judge
applicants based on educational background, years of experience in the
role, and results from a certain aptitude test administered to all
applicants. However, we assume that it is not permissible to judge an
applicant directly on the basis of his/her race.

Let us define the three covariates of interest. The first will be a
binary measure of educational background, \texttt{college} which
indicates whether the applicant graduated college or not. The second,
\texttt{experience} will be the number of years the applicant has worked
in a related role. The third, \texttt{score} is the the applicant's test
score on the aptitude test, which we will suppose is a number between 0
and 100.

We will assume that the first two covariates may depend on the
applicant's race, due to possible socio-economic factors, while the
third has no relationship to the applicant's race.

\begin{Shaded}
\begin{Highlighting}[]
\CommentTok{\# Define the race category, and the three covariates}

\NormalTok{n\_white }\OtherTok{=} \DecValTok{500}
\NormalTok{n\_black }\OtherTok{=} \DecValTok{500}
\NormalTok{n }\OtherTok{=}\NormalTok{ n\_white }\SpecialCharTok{+}\NormalTok{ n\_black}

\NormalTok{race }\OtherTok{=} \FunctionTok{rep}\NormalTok{(}\StringTok{"black"}\NormalTok{, n)}
\NormalTok{race[}\DecValTok{1}\SpecialCharTok{:}\NormalTok{n\_white] }\OtherTok{=} \StringTok{"white"}

\CommentTok{\# bernoulli with higher success probability for whites}
\NormalTok{college }\OtherTok{=} \FunctionTok{rep}\NormalTok{(}\DecValTok{0}\NormalTok{,n)}
\NormalTok{college[race }\SpecialCharTok{==} \StringTok{"white"}\NormalTok{] }\OtherTok{=} \FunctionTok{rbinom}\NormalTok{(n\_white,}\AttributeTok{size =} \DecValTok{1}\NormalTok{, }\AttributeTok{prob =}\NormalTok{ .}\DecValTok{44}\NormalTok{)}
\NormalTok{college[race }\SpecialCharTok{==} \StringTok{"black"}\NormalTok{] }\OtherTok{=} \FunctionTok{rbinom}\NormalTok{(n\_black,}\AttributeTok{size =} \DecValTok{1}\NormalTok{, }\AttributeTok{prob =}\NormalTok{ .}\DecValTok{30}\NormalTok{)}

\CommentTok{\# exponential with higher mean for whites}
\NormalTok{experience }\OtherTok{=} \FunctionTok{rep}\NormalTok{(}\DecValTok{0}\NormalTok{,n)}
\NormalTok{experience[race }\SpecialCharTok{==} \StringTok{"white"}\NormalTok{] }\OtherTok{=} \FunctionTok{round}\NormalTok{(}\FunctionTok{rexp}\NormalTok{(n\_white,}\AttributeTok{rate=}\DecValTok{1}\SpecialCharTok{/}\DecValTok{2}\NormalTok{))}
\NormalTok{experience[race }\SpecialCharTok{==} \StringTok{"black"}\NormalTok{] }\OtherTok{=} \FunctionTok{round}\NormalTok{(}\FunctionTok{rexp}\NormalTok{(n\_black,}\AttributeTok{rate=}\DecValTok{1}\SpecialCharTok{/}\DecValTok{1}\NormalTok{))}

\CommentTok{\# approx normally distributed but cut off at 0 and 100}
\NormalTok{score }\OtherTok{=} \FunctionTok{round}\NormalTok{(}\FunctionTok{pmax}\NormalTok{(}\DecValTok{0}\NormalTok{,}\FunctionTok{pmin}\NormalTok{(}\DecValTok{100}\NormalTok{,}\FunctionTok{rnorm}\NormalTok{(n,}\AttributeTok{mean=}\DecValTok{70}\NormalTok{,}\AttributeTok{sd=}\DecValTok{15}\NormalTok{))),}\DecValTok{2}\NormalTok{)}
\end{Highlighting}
\end{Shaded}

The first case we will consider is the honest employer which does not
take race into account in its application process. Let's call this
Employer A. The response variable, \texttt{hired} will be a binary
variable that records whether an applicant has been hired or not
following the application process. We will imagine that Employer A makes
its hiring decisions using the following idealized method: First, we
randomly generate some (positive) coefficients for each covariate (not
including race). Then, we multiply the (normalized) covariates by the
corresponding coefficients and sum for each applicant. Then we hire the
top ten applicants with the highest totals and reject the rest.

\begin{Shaded}
\begin{Highlighting}[]
\CommentTok{\# First randomly generate some positive coefficients for hiring function}
\NormalTok{coefs }\OtherTok{=} \FunctionTok{runif}\NormalTok{(}\DecValTok{3}\NormalTok{)}

\CommentTok{\# Normalize the covariates }
\NormalTok{norm\_college }\OtherTok{=}\NormalTok{ (college}\SpecialCharTok{{-}}\FunctionTok{mean}\NormalTok{(college))}\SpecialCharTok{/}\FunctionTok{sd}\NormalTok{(college)}
\NormalTok{norm\_experience }\OtherTok{=}\NormalTok{ (experience}\SpecialCharTok{{-}}\FunctionTok{mean}\NormalTok{(experience))}\SpecialCharTok{/}\FunctionTok{sd}\NormalTok{(college) }
\NormalTok{norm\_score }\OtherTok{=}\NormalTok{ (score }\SpecialCharTok{{-}} \FunctionTok{mean}\NormalTok{(score))}\SpecialCharTok{/}\FunctionTok{sd}\NormalTok{(score)}

\CommentTok{\# Compute weighted totals}
\NormalTok{totals }\OtherTok{=}\NormalTok{ coefs[}\DecValTok{1}\NormalTok{]}\SpecialCharTok{*}\NormalTok{norm\_college }\SpecialCharTok{+}\NormalTok{ coefs[}\DecValTok{2}\NormalTok{]}\SpecialCharTok{*}\NormalTok{norm\_experience }\SpecialCharTok{+}\NormalTok{ coefs[}\DecValTok{3}\NormalTok{]}\SpecialCharTok{*}\NormalTok{norm\_score}
\NormalTok{totals[race }\SpecialCharTok{==} \StringTok{"black"}\NormalTok{] }\OtherTok{=}\NormalTok{ totals[race }\SpecialCharTok{==} \StringTok{"black"}\NormalTok{]}

\CommentTok{\# Choose top ten totals and define response variable}
\NormalTok{top.}\FloatTok{10.}\NormalTok{idx }\OtherTok{=} \FunctionTok{order}\NormalTok{(totals,}\AttributeTok{decreasing=}\NormalTok{T)[}\DecValTok{1}\SpecialCharTok{:}\DecValTok{10}\NormalTok{]}
\NormalTok{hired }\OtherTok{=} \FunctionTok{rep}\NormalTok{(}\StringTok{"rejected"}\NormalTok{,n)}
\NormalTok{hired[top.}\FloatTok{10.}\NormalTok{idx] }\OtherTok{=} \StringTok{"hired"}
\end{Highlighting}
\end{Shaded}

Now let us test our pb.fit function on this simulated scenario. We will
assume a binary logistic model.

\begin{Shaded}
\begin{Highlighting}[]
\NormalTok{data }\OtherTok{=} \FunctionTok{data.frame}\NormalTok{(}\AttributeTok{hired =}\NormalTok{ hired,}
                  \AttributeTok{race =}\NormalTok{ race,}
                  \AttributeTok{college =}\NormalTok{ college,}
                  \AttributeTok{experience =}\NormalTok{ experience,}
                  \AttributeTok{score =}\NormalTok{ score)}
\NormalTok{results }\OtherTok{=} \FunctionTok{pb.fit}\NormalTok{(hired }\SpecialCharTok{\textasciitilde{}}\NormalTok{ college }\SpecialCharTok{+}\NormalTok{ experience }\SpecialCharTok{+}\NormalTok{ score,}
                 \AttributeTok{data =}\NormalTok{ data,}
                 \AttributeTok{family =} \StringTok{"multinomial"}\NormalTok{,}
                 \AttributeTok{disparity.group =} \StringTok{"race"}\NormalTok{,}
                 \AttributeTok{majority.group =} \StringTok{"white"}\NormalTok{,}
                 \AttributeTok{minority.group =} \StringTok{"black"}\NormalTok{,}
                 \AttributeTok{base.level =} \StringTok{"hired"}\NormalTok{)}
\FunctionTok{print}\NormalTok{(results)}
\end{Highlighting}
\end{Shaded}

\begin{verbatim}
## $sample.sizes
##           
##            black white
##   hired        1     9
##   rejected   499   491
## 
## $percent.unexplained
##             black
## level=hired  7.23
## 
## $overall.disp
##             black
## level=hired 0.016
## 
## $unexplained.disp
##              black
## level=hired 0.0012
## 
## $unexp.disp.variance
##               black
## level=hired 3.1e-06
\end{verbatim}

\end{document}
