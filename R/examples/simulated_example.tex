% Options for packages loaded elsewhere
\PassOptionsToPackage{unicode}{hyperref}
\PassOptionsToPackage{hyphens}{url}
%
\documentclass[
]{article}
\usepackage{amsmath,amssymb}
\usepackage{lmodern}
\usepackage{ifxetex,ifluatex}
\ifnum 0\ifxetex 1\fi\ifluatex 1\fi=0 % if pdftex
  \usepackage[T1]{fontenc}
  \usepackage[utf8]{inputenc}
  \usepackage{textcomp} % provide euro and other symbols
\else % if luatex or xetex
  \usepackage{unicode-math}
  \defaultfontfeatures{Scale=MatchLowercase}
  \defaultfontfeatures[\rmfamily]{Ligatures=TeX,Scale=1}
\fi
% Use upquote if available, for straight quotes in verbatim environments
\IfFileExists{upquote.sty}{\usepackage{upquote}}{}
\IfFileExists{microtype.sty}{% use microtype if available
  \usepackage[]{microtype}
  \UseMicrotypeSet[protrusion]{basicmath} % disable protrusion for tt fonts
}{}
\makeatletter
\@ifundefined{KOMAClassName}{% if non-KOMA class
  \IfFileExists{parskip.sty}{%
    \usepackage{parskip}
  }{% else
    \setlength{\parindent}{0pt}
    \setlength{\parskip}{6pt plus 2pt minus 1pt}}
}{% if KOMA class
  \KOMAoptions{parskip=half}}
\makeatother
\usepackage{xcolor}
\IfFileExists{xurl.sty}{\usepackage{xurl}}{} % add URL line breaks if available
\IfFileExists{bookmark.sty}{\usepackage{bookmark}}{\usepackage{hyperref}}
\hypersetup{
  pdftitle={Some Examples Using Simulation},
  pdfauthor={Ryan Lafferty},
  hidelinks,
  pdfcreator={LaTeX via pandoc}}
\urlstyle{same} % disable monospaced font for URLs
\usepackage[margin=1in]{geometry}
\usepackage{color}
\usepackage{fancyvrb}
\newcommand{\VerbBar}{|}
\newcommand{\VERB}{\Verb[commandchars=\\\{\}]}
\DefineVerbatimEnvironment{Highlighting}{Verbatim}{commandchars=\\\{\}}
% Add ',fontsize=\small' for more characters per line
\usepackage{framed}
\definecolor{shadecolor}{RGB}{248,248,248}
\newenvironment{Shaded}{\begin{snugshade}}{\end{snugshade}}
\newcommand{\AlertTok}[1]{\textcolor[rgb]{0.94,0.16,0.16}{#1}}
\newcommand{\AnnotationTok}[1]{\textcolor[rgb]{0.56,0.35,0.01}{\textbf{\textit{#1}}}}
\newcommand{\AttributeTok}[1]{\textcolor[rgb]{0.77,0.63,0.00}{#1}}
\newcommand{\BaseNTok}[1]{\textcolor[rgb]{0.00,0.00,0.81}{#1}}
\newcommand{\BuiltInTok}[1]{#1}
\newcommand{\CharTok}[1]{\textcolor[rgb]{0.31,0.60,0.02}{#1}}
\newcommand{\CommentTok}[1]{\textcolor[rgb]{0.56,0.35,0.01}{\textit{#1}}}
\newcommand{\CommentVarTok}[1]{\textcolor[rgb]{0.56,0.35,0.01}{\textbf{\textit{#1}}}}
\newcommand{\ConstantTok}[1]{\textcolor[rgb]{0.00,0.00,0.00}{#1}}
\newcommand{\ControlFlowTok}[1]{\textcolor[rgb]{0.13,0.29,0.53}{\textbf{#1}}}
\newcommand{\DataTypeTok}[1]{\textcolor[rgb]{0.13,0.29,0.53}{#1}}
\newcommand{\DecValTok}[1]{\textcolor[rgb]{0.00,0.00,0.81}{#1}}
\newcommand{\DocumentationTok}[1]{\textcolor[rgb]{0.56,0.35,0.01}{\textbf{\textit{#1}}}}
\newcommand{\ErrorTok}[1]{\textcolor[rgb]{0.64,0.00,0.00}{\textbf{#1}}}
\newcommand{\ExtensionTok}[1]{#1}
\newcommand{\FloatTok}[1]{\textcolor[rgb]{0.00,0.00,0.81}{#1}}
\newcommand{\FunctionTok}[1]{\textcolor[rgb]{0.00,0.00,0.00}{#1}}
\newcommand{\ImportTok}[1]{#1}
\newcommand{\InformationTok}[1]{\textcolor[rgb]{0.56,0.35,0.01}{\textbf{\textit{#1}}}}
\newcommand{\KeywordTok}[1]{\textcolor[rgb]{0.13,0.29,0.53}{\textbf{#1}}}
\newcommand{\NormalTok}[1]{#1}
\newcommand{\OperatorTok}[1]{\textcolor[rgb]{0.81,0.36,0.00}{\textbf{#1}}}
\newcommand{\OtherTok}[1]{\textcolor[rgb]{0.56,0.35,0.01}{#1}}
\newcommand{\PreprocessorTok}[1]{\textcolor[rgb]{0.56,0.35,0.01}{\textit{#1}}}
\newcommand{\RegionMarkerTok}[1]{#1}
\newcommand{\SpecialCharTok}[1]{\textcolor[rgb]{0.00,0.00,0.00}{#1}}
\newcommand{\SpecialStringTok}[1]{\textcolor[rgb]{0.31,0.60,0.02}{#1}}
\newcommand{\StringTok}[1]{\textcolor[rgb]{0.31,0.60,0.02}{#1}}
\newcommand{\VariableTok}[1]{\textcolor[rgb]{0.00,0.00,0.00}{#1}}
\newcommand{\VerbatimStringTok}[1]{\textcolor[rgb]{0.31,0.60,0.02}{#1}}
\newcommand{\WarningTok}[1]{\textcolor[rgb]{0.56,0.35,0.01}{\textbf{\textit{#1}}}}
\usepackage{graphicx}
\makeatletter
\def\maxwidth{\ifdim\Gin@nat@width>\linewidth\linewidth\else\Gin@nat@width\fi}
\def\maxheight{\ifdim\Gin@nat@height>\textheight\textheight\else\Gin@nat@height\fi}
\makeatother
% Scale images if necessary, so that they will not overflow the page
% margins by default, and it is still possible to overwrite the defaults
% using explicit options in \includegraphics[width, height, ...]{}
\setkeys{Gin}{width=\maxwidth,height=\maxheight,keepaspectratio}
% Set default figure placement to htbp
\makeatletter
\def\fps@figure{htbp}
\makeatother
\setlength{\emergencystretch}{3em} % prevent overfull lines
\providecommand{\tightlist}{%
  \setlength{\itemsep}{0pt}\setlength{\parskip}{0pt}}
\setcounter{secnumdepth}{-\maxdimen} % remove section numbering
\ifluatex
  \usepackage{selnolig}  % disable illegal ligatures
\fi

\title{Some Examples Using Simulation}
\author{Ryan Lafferty}
\date{2/28/2022}

\begin{document}
\maketitle

\hypertarget{simulated-racial-discrimination}{%
\subsubsection{Simulated Racial
Discrimination}\label{simulated-racial-discrimination}}

Let's try to simulate a situation where discrimination is happening and
see if the PB method can detect it. Suppose black and white applicants
are applying for a certain job. Imagine that it is permissible to judge
applicants based on educational background, years of experience in the
role, and results from a certain aptitude test administered to all
applicants. However, we assume that it is not permissible to judge an
applicant directly on the basis of his/her race.

Let us define the three covariates of interest. The first will be a
binary measure of educational background, \texttt{college} which
indicates whether the applicant graduated college or not. The second,
\texttt{experience} will be the number of years the applicant has worked
in a related role. The third, \texttt{score} is the the applicant's test
score on the aptitude test, which we will suppose is a number between 0
and 100.

We will assume that the first two covariates may depend on the
applicant's race, due to possible socio-economic factors, while the
third has no relationship to the applicant's race.

\begin{Shaded}
\begin{Highlighting}[]
\CommentTok{\# Define the race category, and the three covariates}

\NormalTok{n\_white }\OtherTok{=} \DecValTok{50}
\NormalTok{n\_black }\OtherTok{=} \DecValTok{50}
\NormalTok{n }\OtherTok{=}\NormalTok{ n\_white }\SpecialCharTok{+}\NormalTok{ n\_black}

\NormalTok{race }\OtherTok{=} \FunctionTok{rep}\NormalTok{(}\StringTok{"black"}\NormalTok{, n)}
\NormalTok{race[}\DecValTok{1}\SpecialCharTok{:}\NormalTok{n\_white] }\OtherTok{=} \StringTok{"white"}

\CommentTok{\# bernoulli with higher success probability for whites}
\NormalTok{college }\OtherTok{=} \FunctionTok{rep}\NormalTok{(}\DecValTok{0}\NormalTok{,n)}
\NormalTok{college[race }\SpecialCharTok{==} \StringTok{"white"}\NormalTok{] }\OtherTok{=} \FunctionTok{rbinom}\NormalTok{(n\_white,}\AttributeTok{size =} \DecValTok{1}\NormalTok{, }\AttributeTok{prob =}\NormalTok{ .}\DecValTok{44}\NormalTok{)}
\NormalTok{college[race }\SpecialCharTok{==} \StringTok{"black"}\NormalTok{] }\OtherTok{=} \FunctionTok{rbinom}\NormalTok{(n\_black,}\AttributeTok{size =} \DecValTok{1}\NormalTok{, }\AttributeTok{prob =}\NormalTok{ .}\DecValTok{30}\NormalTok{)}

\CommentTok{\# exponential with higher mean for whites}
\NormalTok{experience }\OtherTok{=} \FunctionTok{rep}\NormalTok{(}\DecValTok{0}\NormalTok{,n)}
\NormalTok{experience[race }\SpecialCharTok{==} \StringTok{"white"}\NormalTok{] }\OtherTok{=} \FunctionTok{round}\NormalTok{(}\FunctionTok{rexp}\NormalTok{(n\_white,}\AttributeTok{rate=}\DecValTok{1}\SpecialCharTok{/}\FloatTok{1.5}\NormalTok{))}
\NormalTok{experience[race }\SpecialCharTok{==} \StringTok{"black"}\NormalTok{] }\OtherTok{=} \FunctionTok{round}\NormalTok{(}\FunctionTok{rexp}\NormalTok{(n\_black,}\AttributeTok{rate=}\DecValTok{1}\SpecialCharTok{/}\DecValTok{1}\NormalTok{))}

\CommentTok{\# approx normally distributed but cut off at 0 and 100}
\NormalTok{score }\OtherTok{=} \FunctionTok{round}\NormalTok{(}\FunctionTok{pmax}\NormalTok{(}\DecValTok{0}\NormalTok{,}\FunctionTok{pmin}\NormalTok{(}\DecValTok{100}\NormalTok{,}\FunctionTok{rnorm}\NormalTok{(n,}\AttributeTok{mean=}\DecValTok{70}\NormalTok{,}\AttributeTok{sd=}\DecValTok{15}\NormalTok{))),}\DecValTok{2}\NormalTok{)}
\end{Highlighting}
\end{Shaded}

The first case we will consider is the honest employer which does not
take race into account in its application process. Let's call this
Employer A. The response variable, \texttt{hired} will be a binary
variable that records whether an applicant has been hired or not
following the application process. We will imagine that Employer A makes
its hiring decisions using the following idealized method: First, we
randomly generate some (positive) coefficients for each covariate (not
including race). Then, we multiply the (normalized) covariates by the
corresponding coefficients and sum for each applicant. Then we hire the
top ten applicants with the highest totals and reject the rest.

\begin{Shaded}
\begin{Highlighting}[]
 \CommentTok{\# First randomly generate some positive coefficients for hiring function}
\NormalTok{ coefs }\OtherTok{=} \FunctionTok{runif}\NormalTok{(}\DecValTok{3}\NormalTok{)}
 
 \CommentTok{\# Normalize the covariates }
\NormalTok{ norm\_college }\OtherTok{=}\NormalTok{ (college}\SpecialCharTok{{-}}\FunctionTok{mean}\NormalTok{(college))}\SpecialCharTok{/}\FunctionTok{sd}\NormalTok{(college)}
\NormalTok{ norm\_experience }\OtherTok{=}\NormalTok{ (experience}\SpecialCharTok{{-}}\FunctionTok{mean}\NormalTok{(experience))}\SpecialCharTok{/}\FunctionTok{sd}\NormalTok{(college) }
\NormalTok{ norm\_score }\OtherTok{=}\NormalTok{ (score }\SpecialCharTok{{-}} \FunctionTok{mean}\NormalTok{(score))}\SpecialCharTok{/}\FunctionTok{sd}\NormalTok{(score)}
 
 \CommentTok{\# Compute weighted totals}
\NormalTok{ totals }\OtherTok{=}\NormalTok{ coefs[}\DecValTok{1}\NormalTok{]}\SpecialCharTok{*}\NormalTok{norm\_college }\SpecialCharTok{+}\NormalTok{ coefs[}\DecValTok{2}\NormalTok{]}\SpecialCharTok{*}\NormalTok{norm\_experience }\SpecialCharTok{+}\NormalTok{ coefs[}\DecValTok{3}\NormalTok{]}\SpecialCharTok{*}\NormalTok{norm\_score}
 
 \CommentTok{\# Choose top ten totals and define response variable}
\NormalTok{ top.}\FloatTok{10.}\NormalTok{idx }\OtherTok{=} \FunctionTok{order}\NormalTok{(totals,}\AttributeTok{decreasing=}\NormalTok{T)[}\DecValTok{1}\SpecialCharTok{:}\DecValTok{10}\NormalTok{]}
\NormalTok{ hired }\OtherTok{=} \FunctionTok{rep}\NormalTok{(}\StringTok{"rejected"}\NormalTok{,n)}
\NormalTok{ hired[top.}\FloatTok{10.}\NormalTok{idx] }\OtherTok{=} \StringTok{"hired"}
\end{Highlighting}
\end{Shaded}

Now let us test our pb.fit function on this simulated scenario. We will
assume a binary logistic model.

\begin{Shaded}
\begin{Highlighting}[]
\NormalTok{ data }\OtherTok{=} \FunctionTok{data.frame}\NormalTok{(}\AttributeTok{hired =}\NormalTok{ hired,}
                   \AttributeTok{race =}\NormalTok{ race,}
                   \AttributeTok{college =}\NormalTok{ college,}
                   \AttributeTok{experience =}\NormalTok{ experience,}
                   \AttributeTok{score =}\NormalTok{ score)}
\NormalTok{ results }\OtherTok{=} \FunctionTok{pb.fit}\NormalTok{(hired }\SpecialCharTok{\textasciitilde{}}\NormalTok{ college }\SpecialCharTok{+}\NormalTok{ experience }\SpecialCharTok{+}\NormalTok{ score,}
                  \AttributeTok{data =}\NormalTok{ data,}
                  \AttributeTok{family =} \StringTok{"multinomial"}\NormalTok{,}
                  \AttributeTok{disparity.group =} \StringTok{"race"}\NormalTok{,}
                  \AttributeTok{majority.group =} \StringTok{"white"}\NormalTok{,}
                  \AttributeTok{minority.group =} \StringTok{"black"}\NormalTok{,}
                  \AttributeTok{base.level =} \StringTok{"hired"}\NormalTok{)}
 \FunctionTok{print}\NormalTok{(results)}
\end{Highlighting}
\end{Shaded}

\begin{verbatim}
## $reference.model
## Call:
## nnet::multinom(formula = factor_y[deltaR0] ~ cov.x[deltaR0, ], 
##     weights = w[deltaR0], trace = F)
## 
## Coefficients:
##                (Intercept)    cov.x[deltaR0, ]college 
##                 53.7664732                -20.0356323 
## cov.x[deltaR0, ]experience      cov.x[deltaR0, ]score 
##                 -8.4280023                 -0.2299871 
## 
## Residual Deviance: 0.1000077 
## AIC: 8.100008 
## 
## $sample.sizes
##           
##            black white
##   hired        5     5
##   rejected    45    45
## 
## $observed.estimate
##             black
## level=hired   0.1
## 
## $percent.unexplained
##             black
## level=hired  -Inf
## 
## $overall.disp
##             black
## level=hired     0
## 
## $unexplained.disp
##              black
## level=hired -0.043
## 
## $unexp.disp.variance
##               black
## level=hired 0.00074
\end{verbatim}

We can use the results for unexplained disparity and unexplained
disparity variance to perform a simple asymptotic test to check for
significance.

\begin{Shaded}
\begin{Highlighting}[]
 \CommentTok{\#test H0: unexplained.disp = 0}
 \CommentTok{\#vs.  H1: unexplained.disp \textgreater{} 0}
 
\NormalTok{ zscore }\OtherTok{=}\NormalTok{ results}\SpecialCharTok{$}\NormalTok{unexplained.disp}\SpecialCharTok{/}\FunctionTok{sqrt}\NormalTok{(results}\SpecialCharTok{$}\NormalTok{unexp.disp.variance)}
 
 \FunctionTok{print}\NormalTok{(}\FunctionTok{paste}\NormalTok{(}\StringTok{"P{-}value:"}\NormalTok{, }\FunctionTok{pnorm}\NormalTok{(zscore[[}\DecValTok{1}\NormalTok{]],}\AttributeTok{lower.tail=}\ConstantTok{FALSE}\NormalTok{)))}
\end{Highlighting}
\end{Shaded}

\begin{verbatim}
## [1] "P-value: 0.943027983996444"
\end{verbatim}

Now let us perform the same analysis on Employer B. We will assume
Employer B is deliberately discriminating against black applicants by
cutting \texttt{totals} for black applicants in half before ranking
them.

\begin{Shaded}
\begin{Highlighting}[]
 \CommentTok{\# Cut totals in half for black applicants}
 
\NormalTok{ totals.A }\OtherTok{=}\NormalTok{ totals}
\NormalTok{ totals.B }\OtherTok{=}\NormalTok{ totals}
\NormalTok{ totals.B[race }\SpecialCharTok{==} \StringTok{"black"}\NormalTok{] }\OtherTok{=}\NormalTok{ totals[race }\SpecialCharTok{==} \StringTok{"black"}\NormalTok{]}\SpecialCharTok{/}\DecValTok{2}
 
 \CommentTok{\# Choose top ten totals and define response variable}
\NormalTok{ top.}\FloatTok{10.}\NormalTok{idx }\OtherTok{=} \FunctionTok{order}\NormalTok{(totals.B,}\AttributeTok{decreasing=}\NormalTok{T)[}\DecValTok{1}\SpecialCharTok{:}\DecValTok{10}\NormalTok{]}
\NormalTok{ hired }\OtherTok{=} \FunctionTok{rep}\NormalTok{(}\StringTok{"rejected"}\NormalTok{,n)}
\NormalTok{ hired[top.}\FloatTok{10.}\NormalTok{idx] }\OtherTok{=} \StringTok{"hired"}
 
\NormalTok{ data }\OtherTok{=} \FunctionTok{data.frame}\NormalTok{(}\AttributeTok{hired =}\NormalTok{ hired,}
                   \AttributeTok{race =}\NormalTok{ race,}
                   \AttributeTok{college =}\NormalTok{ college,}
                   \AttributeTok{experience =}\NormalTok{ experience,}
                   \AttributeTok{score =}\NormalTok{ score)}
\NormalTok{ results }\OtherTok{=} \FunctionTok{pb.fit}\NormalTok{(hired }\SpecialCharTok{\textasciitilde{}}\NormalTok{ college }\SpecialCharTok{+}\NormalTok{ experience }\SpecialCharTok{+}\NormalTok{ score,}
                  \AttributeTok{data =}\NormalTok{ data,}
                  \AttributeTok{family =} \StringTok{"multinomial"}\NormalTok{,}
                  \AttributeTok{disparity.group =} \StringTok{"race"}\NormalTok{,}
                  \AttributeTok{majority.group =} \StringTok{"white"}\NormalTok{,}
                  \AttributeTok{minority.group =} \StringTok{"black"}\NormalTok{,}
                  \AttributeTok{base.level =} \StringTok{"hired"}\NormalTok{)}
 \FunctionTok{print}\NormalTok{(results)}
\end{Highlighting}
\end{Shaded}

\begin{verbatim}
## $reference.model
## Call:
## nnet::multinom(formula = factor_y[deltaR0] ~ cov.x[deltaR0, ], 
##     weights = w[deltaR0], trace = F)
## 
## Coefficients:
##                (Intercept)    cov.x[deltaR0, ]college 
##                 89.1101858                -35.2033928 
## cov.x[deltaR0, ]experience      cov.x[deltaR0, ]score 
##                -23.2222025                 -0.2854494 
## 
## Residual Deviance: 0.1343559 
## AIC: 8.134356 
## 
## $sample.sizes
##           
##            black white
##   hired        1     9
##   rejected    49    41
## 
## $observed.estimate
##             black
## level=hired  0.02
## 
## $percent.unexplained
##             black
## level=hired  62.7
## 
## $overall.disp
##             black
## level=hired  0.16
## 
## $unexplained.disp
##             black
## level=hired   0.1
## 
## $unexp.disp.variance
##              black
## level=hired 0.0018
\end{verbatim}

Now let's perform the test again and see if we notice any difference in
the p-value.

\begin{Shaded}
\begin{Highlighting}[]
 \CommentTok{\#test H0: unexplained.disp = 0}
 \CommentTok{\#vs.  H1: unexplained.disp \textgreater{} 0}
 
\NormalTok{ zscore }\OtherTok{=}\NormalTok{ results}\SpecialCharTok{$}\NormalTok{unexplained.disp}\SpecialCharTok{/}\FunctionTok{sqrt}\NormalTok{(results}\SpecialCharTok{$}\NormalTok{unexp.disp.variance)}
 
 \FunctionTok{print}\NormalTok{(}\FunctionTok{paste}\NormalTok{(}\StringTok{"P{-}value:"}\NormalTok{, }\FunctionTok{pnorm}\NormalTok{(zscore[[}\DecValTok{1}\NormalTok{]],}\AttributeTok{lower.tail=}\ConstantTok{FALSE}\NormalTok{)))}
\end{Highlighting}
\end{Shaded}

\begin{verbatim}
## [1] "P-value: 0.0092110627270495"
\end{verbatim}

We need to be very careful when interpreting the results of our
analysis, especially in a high stakes scenario such as a legal battle.
Neglecting to take adequate care could result in failing to render
justice to a guilty party or falsely accusing an innocent party. Model
assumptions are critical and can heavily influence the outcome of the
analysis. In particular, we must be reasonably sure that there do not
exist unobserved covariates that affect the hiring decision, but which
would be legitimate to use in selecting the best candidate. In this
simple model we have the ability to know exactly which variables are
used in making hiring decisions. However, in the real world this can be
more difficult.

\hypertarget{ppo-simulation}{%
\subsubsection{PPO Simulation}\label{ppo-simulation}}

Now let's consider a healthcare scenario. It has been reported that
white patients are often perceived as having a lower pain tolerance than
black patients, and as a result, may be given more frequent or higher
doses of pain medications. Let us imagine an artificial scenario where
we can apply the \texttt{pb.fit} method to test whether discrimination
is occurring.

Suppose we have 500 white patients and 500 black patients having pain
symptoms. These symptoms may be the result of various diseases. Perhaps
disease A is more common in the white population, and disease B is more
common in the black population. Hence, if disease A tends to elicit more
pain symptoms than disease B, it is plausible that average assessed pain
levels are actually less in the black patients than in the white
patients. Thus, to detect discrimination we will have to do a PB
analysis.

Let us assume two covariates. First, we consider \texttt{disease\_type}
which is a binary variable, taking value 0 if the patient has (known to
be more painful) disease A and value 1 if the patient has (known to be
less painful) disease B. In addition, we may consider a patient's
reported pain level, \texttt{reported}, between 0 and 10, which we will
assume not to be directly affected by the patient's race.

\begin{Shaded}
\begin{Highlighting}[]
\CommentTok{\# Define the race category, and the two covariates}

\NormalTok{n\_white }\OtherTok{=} \DecValTok{500}
\NormalTok{n\_black }\OtherTok{=} \DecValTok{500}
\NormalTok{n }\OtherTok{=}\NormalTok{ n\_white }\SpecialCharTok{+}\NormalTok{ n\_black}

\NormalTok{race }\OtherTok{=} \FunctionTok{rep}\NormalTok{(}\StringTok{"black"}\NormalTok{, n)}
\NormalTok{race[}\DecValTok{1}\SpecialCharTok{:}\NormalTok{n\_white] }\OtherTok{=} \StringTok{"white"}

\CommentTok{\# bernoulli with higher success probability for black patients}
\NormalTok{disease\_type }\OtherTok{=} \FunctionTok{rep}\NormalTok{(}\DecValTok{0}\NormalTok{,n)}
\NormalTok{disease\_type[race }\SpecialCharTok{==} \StringTok{"white"}\NormalTok{] }\OtherTok{=} \FunctionTok{rbinom}\NormalTok{(n\_white,}\AttributeTok{size =} \DecValTok{1}\NormalTok{, }\AttributeTok{prob =}\NormalTok{ .}\DecValTok{4}\NormalTok{)}
\NormalTok{disease\_type[race }\SpecialCharTok{==} \StringTok{"black"}\NormalTok{] }\OtherTok{=} \FunctionTok{rbinom}\NormalTok{(n\_black,}\AttributeTok{size =} \DecValTok{1}\NormalTok{, }\AttributeTok{prob =}\NormalTok{ .}\DecValTok{6}\NormalTok{)}

\CommentTok{\# reported should be higher for patients with disease B}
\NormalTok{reported }\OtherTok{=} \FunctionTok{rep}\NormalTok{(}\DecValTok{0}\NormalTok{,n)}
\NormalTok{reported[disease\_type }\SpecialCharTok{==} \DecValTok{0}\NormalTok{] }\OtherTok{=} \FunctionTok{round}\NormalTok{(}\FunctionTok{runif}\NormalTok{(}\FunctionTok{sum}\NormalTok{(disease\_type }\SpecialCharTok{==} \DecValTok{0}\NormalTok{),}\DecValTok{0}\NormalTok{,}\DecValTok{10}\NormalTok{))}
\NormalTok{reported[disease\_type }\SpecialCharTok{==} \DecValTok{1}\NormalTok{] }\OtherTok{=} \FunctionTok{round}\NormalTok{(}\FunctionTok{runif}\NormalTok{(}\FunctionTok{sum}\NormalTok{(disease\_type }\SpecialCharTok{==} \DecValTok{1}\NormalTok{),}\DecValTok{0}\NormalTok{,}\DecValTok{10}\NormalTok{))}
\end{Highlighting}
\end{Shaded}

Let us imagine that a biased doctor is estimating the patient's true
pain level on a scale from 0 to 10 based on the covariates given.
Suppose the doctor uses a linear combination of the covariates, together
with race, to make a decision.

\begin{Shaded}
\begin{Highlighting}[]
 \CommentTok{\# Compute a pain score based on disease type and reported pain level + noise}
\NormalTok{ score }\OtherTok{=}\NormalTok{ (}\DecValTok{2}\SpecialCharTok{*}\NormalTok{disease\_type }\SpecialCharTok{+}\NormalTok{ reported }\SpecialCharTok{+} \FunctionTok{rnorm}\NormalTok{(n))}\SpecialCharTok{/}\DecValTok{12}
\NormalTok{ score }\OtherTok{=}\NormalTok{ score}\SpecialCharTok{*}\DecValTok{10}
\NormalTok{ score }\OtherTok{=} \FunctionTok{pmax}\NormalTok{(}\DecValTok{0}\NormalTok{,}\FunctionTok{pmin}\NormalTok{(}\DecValTok{10}\NormalTok{,score))}
 
 \CommentTok{\# for black patients, multiply score by a factor of J}
\NormalTok{ J }\OtherTok{=}\NormalTok{ .}\DecValTok{5}
\NormalTok{ score[race }\SpecialCharTok{==} \StringTok{"black"}\NormalTok{] }\OtherTok{=}\NormalTok{ score[race }\SpecialCharTok{==} \StringTok{"black"}\NormalTok{]}\SpecialCharTok{*}\NormalTok{J}
 
 \CommentTok{\# Round the scores to get an assessed pain level}
\NormalTok{ assessed }\OtherTok{=} \FunctionTok{round}\NormalTok{(score)}
\end{Highlighting}
\end{Shaded}

Now let us test our pb.fit function on this simulated scenario. We will
assume a proportional odds model.

\begin{Shaded}
\begin{Highlighting}[]
\NormalTok{ data }\OtherTok{=} \FunctionTok{data.frame}\NormalTok{(}\AttributeTok{assessed =}\NormalTok{ assessed,}
                   \AttributeTok{race =}\NormalTok{ race,}
                   \AttributeTok{disease\_type =}\NormalTok{ disease\_type,}
                   \AttributeTok{reported =}\NormalTok{ reported)}
\NormalTok{ results }\OtherTok{=} \FunctionTok{pb.fit}\NormalTok{(assessed }\SpecialCharTok{\textasciitilde{}}\NormalTok{ disease\_type }\SpecialCharTok{+}\NormalTok{ reported,}
                  \AttributeTok{data =}\NormalTok{ data,}
                  \AttributeTok{family =} \StringTok{"ordinal"}\NormalTok{,}
                  \AttributeTok{disparity.group =} \StringTok{"race"}\NormalTok{,}
                  \AttributeTok{majority.group =} \StringTok{"white"}\NormalTok{,}
                  \AttributeTok{minority.group =} \StringTok{"black"}\NormalTok{,}
                  \AttributeTok{prop.odds.fail =} \FunctionTok{c}\NormalTok{(}\StringTok{"reported"}\NormalTok{))}
 \FunctionTok{print}\NormalTok{(results)}
\end{Highlighting}
\end{Shaded}

\begin{verbatim}
## $reference.model
## formula: as.factor(y[deltaR0]) ~ cov.x[deltaR0, ]
## nominal: ~cov.z[deltaR0, ]
## 
##  link  threshold nobs logLik  AIC     niter max.grad cond.H 
##  logit flexible  500  -555.90 1153.80 8(0)  2.42e-13 4.0e+04
## 
## Coefficients:
## cov.x[deltaR0, ] 
##            3.767 
## 
## Threshold coefficients:
##                  0|1    1|2    2|3    3|4    4|5    5|6    6|7    7|8    8|9   
## (Intercept)       1.624  3.934  5.495  9.028 11.067 14.957 16.083 16.598 19.134
## cov.z[deltaR0, ] -2.337 -2.661 -1.850 -2.085 -2.105 -2.305 -2.068 -1.836 -1.878
##                  9|10  
## (Intercept)      26.214
## cov.z[deltaR0, ] -2.459
## 
## $sample.sizes
##     
##      black white
##   0     42    27
##   1     89    32
##   10     0    22
##   2    115    61
##   3    125    68
##   4     88    49
##   5     41    66
##   6      0    59
##   7      0    51
##   8      0    41
##   9      0    24
## 
## $observed.estimate
##         black
## level=0 0.084
## level=1 0.180
## level=2 0.230
## level=3 0.250
## level=4 0.180
## level=5 0.082
## level=6 0.000
## level=7 0.000
## level=8 0.000
## level=9 0.000
## 
## $percent.unexplained
##          black
## level=0 114.51
## level=1 109.07
## level=2 115.79
## level=3 106.68
## level=4 114.64
## level=5 128.62
## level=6 108.80
## level=7 102.75
## level=8 112.57
## level=9 121.02
## 
## $overall.disp
##          black
## level=0 -0.030
## level=1 -0.110
## level=2 -0.110
## level=3 -0.110
## level=4 -0.078
## level=5  0.050
## level=6  0.120
## level=7  0.100
## level=8  0.082
## level=9  0.048
## 
## $unexplained.disp
##          black
## level=0 -0.034
## level=1 -0.120
## level=2 -0.130
## level=3 -0.120
## level=4 -0.089
## level=5  0.064
## level=6  0.130
## level=7  0.100
## level=8  0.092
## level=9  0.058
## 
## $unexp.disp.variance
##           black
## level=0 0.00012
## level=1 0.00033
## level=2 0.00053
## level=3 0.00068
## level=4 0.00050
## level=5 0.00047
## level=6 0.00026
## level=7 0.00020
## level=8 0.00018
## level=9 0.00013
\end{verbatim}

\end{document}
